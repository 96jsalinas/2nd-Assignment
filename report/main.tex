% Main Report - Advanced Programming 2nd Assignment
% Bank Marketing Dataset - Classification Analysis
% Student ID: 100562234

\documentclass[11pt,a4paper]{article}

% Packages
\usepackage[utf8]{inputenc}
\usepackage[T1]{fontenc}
\usepackage{lmodern}
\usepackage[margin=2.5cm]{geometry}
\usepackage{graphicx}
\usepackage{booktabs}
\usepackage{longtable}
\usepackage{amsmath}
\usepackage{hyperref}
\usepackage{listings}
\usepackage{xcolor}
\usepackage{float}
\usepackage{caption}
\usepackage{subcaption}

% Code listing style
\lstset{
    language=Python,
    basicstyle=\ttfamily\small,
    keywordstyle=\color{blue},
    stringstyle=\color{red},
    commentstyle=\color{green!50!black},
    numbers=left,
    numberstyle=\tiny\color{gray},
    stepnumber=1,
    numbersep=5pt,
    backgroundcolor=\color{gray!10},
    showspaces=false,
    showstringspaces=false,
    frame=single,
    breaklines=true
}

% Hyperref setup
\hypersetup{
    colorlinks=true,
    linkcolor=blue,
    filecolor=magenta,
    urlcolor=cyan,
}

% Title information
\title{Advanced Programming - 2nd Assignment\\
\large Bank Marketing Dataset Classification Analysis}
\author{Josu Salinas Colina}
\date{\today}

\begin{document}

\maketitle

% ============================================================================
\section{Introduction}
% ============================================================================

\subsection{Project Overview}
The objective of this project is to analyze the Bank Marketing dataset to predict whether a client will subscribe to a term deposit. The dataset consists of data from direct marketing campaigns of a Portuguese banking institution. This analysis is performed in the context of the Advanced Programming assignment, using Student ID 100562234 as the random seed.

\subsection{Use of AI}
\subsubsection{AI-Assisted Project Scaffolding}
The initial structure of this project was created using Claude Opus 4.5 (Thinking), an advanced AI assistant. The process followed these steps:

\begin{enumerate}
    \item \textbf{Requirements Documentation:} A detailed \texttt{requirements.MD} file was first created manually, capturing all the specific requirements for the assignment, including the five main tasks, evaluation metrics, preprocessing guidelines, and general project constraints.
    
    \item \textbf{AI-Assisted Scaffolding:} Using a conversational interface with Claude Opus 4.5, the AI was provided with the requirements document and the existing data folder structure. The explicit instruction was to create only the barebones project structure---placeholder files with proper organization and TODO markers---without implementing any actual functionality.
    
    \item \textbf{Generated Structure:} The AI created the complete project scaffold, including:
    \begin{itemize}
        \item Python source files (\texttt{src/}) with placeholder functions and docstrings
        \item \LaTeX{} report files (\texttt{report/}) with section templates
        \item Supporting files such as \texttt{README.md} and \texttt{.gitignore}
        \item Directory structure for outputs and figures
    \end{itemize}
\end{enumerate}

\subsubsection{AI-Assisted Code review and documentation}
After the code was manually developed the same conversational interface was used to ask for a review of the code, validating the changes and checking for suggestions. Finally, after the code was deemed complete and correct, the same system was used to generate code comments and descriptions of the whole project. This was particularly useful for tedious tasks such as proper formatting of outputs. AI was also used to aid in generating the code that generates the graph of the decision tree that is used in this report.

\subsubsection{AI review of report}
This same report was provided to the same AI assistant to validate the report and check for suggestions, the source \LaTeX code was provided in the same repository as the whole project so that the AI assistant could have access to the whole context, including the ability to read the output of the python script.


% ============================================================================
% Include task-specific sections
% ============================================================================
% Task 1: Simplified Exploratory Data Analysis
% ============================================================================
\section{Task 1: Simplified Exploratory Data Analysis}
% ============================================================================

This section presents the results of a simplified Exploratory Data Analysis (EDA) performed on the Bank Marketing dataset.

\subsection{Dataset Overview}

The Bank Marketing dataset contains information about direct marketing campaigns (phone calls) of a Portuguese banking institution. The goal is to predict whether a client will subscribe to a term deposit.

\begin{table}[H]
\centering
\caption{Dataset Dimensions}
\begin{tabular}{lr}
\toprule
\textbf{Metric} & \textbf{Value} \\
\midrule
Number of instances & 11,000 \\
Number of features & 16 \\
Target variable & \texttt{deposit} \\
\bottomrule
\end{tabular}
\end{table}

\subsection{Variable Types}

Variables were classified based on their data types. Columns with \texttt{object} type are considered categorical, while numeric types (\texttt{int64}, \texttt{float64}) are considered numerical.

\subsubsection{Numerical Variables (7)}
\begin{itemize}
    \item \texttt{age} -- age in years
    \item \texttt{balance} -- average yearly balance
    \item \texttt{day} -- last contact day
    \item \texttt{duration} -- last contact duration in seconds
    \item \texttt{campaign} -- number of contacts during this campaign
    \item \texttt{pdays} -- days since last contact from previous campaign
    \item \texttt{previous} -- number of contacts before this campaign
\end{itemize}

\subsubsection{Categorical Variables (10)}
\begin{itemize}
    \item \texttt{job} -- type of job (12 unique values)
    \item \texttt{marital} -- marital status (3 unique values)
    \item \texttt{education} -- education level (4 unique values)
    \item \texttt{default} -- has credit in default (2 unique values)
    \item \texttt{housing} -- has housing loan (2 unique values)
    \item \texttt{loan} -- has personal loan (2 unique values)
    \item \texttt{contact} -- contact communication type (3 unique values)
    \item \texttt{month} -- last contact month (12 unique values)
    \item \texttt{poutcome} -- outcome of previous campaign (4 unique values)
    \item \texttt{deposit} -- target variable (2 unique values)
\end{itemize}

\subsubsection{High Cardinality Variables}

Using a threshold of 10 unique values, the following categorical variables exhibit high cardinality:

\begin{table}[H]
\centering
\caption{High Cardinality Categorical Variables}
\begin{tabular}{lr}
\toprule
\textbf{Variable} & \textbf{Unique Values} \\
\midrule
\texttt{job} & 12 \\
\texttt{month} & 12 \\
\bottomrule
\end{tabular}
\end{table}

These variables may require special encoding strategies (e.g., target encoding) in preprocessing, though standard one-hot encoding remains viable given the moderate cardinality.

\subsection{Missing Values Analysis}

\begin{table}[H]
\centering
\caption{Columns with Missing Values}
\begin{tabular}{lrr}
\toprule
\textbf{Column} & \textbf{Missing Count} & \textbf{Percentage} \\
\midrule
\texttt{job} & 327 & 2.97\% \\
\bottomrule
\end{tabular}
\end{table}

Only the \texttt{job} column contains missing values, representing approximately 3\% of the data. This will require imputation during preprocessing, where the most frequent category or a separate ``missing'' category can be used.

\subsection{Constant and ID Columns}

The analysis checked for:
\begin{itemize}
    \item \textbf{Constant columns}: Columns where all values are identical (useless for prediction)
    \item \textbf{Potential ID columns}: Columns where all values are unique (likely identifiers)
\end{itemize}

\textbf{Result}: No constant columns and no potential ID columns were found in the dataset. All features contain meaningful variation.

\subsection{Problem Type Identification}

\begin{table}[H]
\centering
\caption{Target Variable Analysis}
\begin{tabular}{lr}
\toprule
\textbf{Property} & \textbf{Value} \\
\midrule
Target variable & \texttt{deposit} \\
Data type & object (categorical) \\
Unique values & 2 (yes, no) \\
Problem type & \textbf{Binary Classification} \\
\bottomrule
\end{tabular}
\end{table}

\subsection{Class Imbalance}

\begin{table}[H]
\centering
\caption{Class Distribution}
\begin{tabular}{lrr}
\toprule
\textbf{Class} & \textbf{Count} & \textbf{Percentage} \\
\midrule
no & 5,780 & 52.55\% \\
yes & 5,220 & 47.45\% \\
\midrule
\textbf{Imbalance ratio} & \multicolumn{2}{c}{1.11} \\
\bottomrule
\end{tabular}
\end{table}

The dataset is \textbf{not imbalanced}. With an imbalance ratio of 1.11 (using a threshold of 1.5), the classes are well-balanced. This means we can use accuracy as an appropriate evaluation metric without requiring techniques like oversampling or class weighting.

\subsection{Special Variable: pdays}

The \texttt{pdays} variable requires special attention as it encodes two types of information:
\begin{itemize}
    \item \textbf{-1}: Client was not contacted in a previous campaign (or unknown)
    \item \textbf{$\geq$ 0}: Number of days since the client was last contacted
\end{itemize}

\begin{table}[H]
\centering
\caption{pdays Analysis}
\begin{tabular}{lr}
\toprule
\textbf{Metric} & \textbf{Value} \\
\midrule
Total observations & 11,000 \\
Values = -1 (no previous contact) & 8,203 (74.57\%) \\
Values $>$ -1 (previous contact) & 2,797 (25.43\%) \\
\midrule
\multicolumn{2}{c}{\textit{Statistics for contacted clients only:}} \\
Minimum & 1 days \\
Maximum & 854 days \\
Mean & 204.72 days \\
Median & 182 days \\
\bottomrule
\end{tabular}
\end{table}

\subsubsection{Preprocessing Alternatives}

Three preprocessing strategies were considered for handling \texttt{pdays}:

\begin{enumerate}
    \item \textbf{Option A: Median Imputation}
    \begin{itemize}
        \item Replace -1 values with the median of valid pdays values (182 days)
        \item \textit{Pros}: Simple, maintains single feature
        \item \textit{Cons}: Loses information about whether contact occurred; may introduce misleading patterns
    \end{itemize}
    
    \item \textbf{Option B: Large Value Replacement}
    \begin{itemize}
        \item Replace -1 with a large value (e.g., 999 or max+1 = 855)
        \item \textit{Pros}: Simple, maintains single feature, distinguishes no-contact cases
        \item \textit{Cons}: Arbitrary choice of replacement value; may affect distance-based methods
    \end{itemize}
    
    \item \textbf{Option C: Two-Feature Approach (Selected)}
    \begin{itemize}
        \item Create binary indicator: \texttt{was\_contacted\_before} = 1 if pdays $\neq$ -1, else 0
        \item Keep transformed \texttt{pdays} column (with -1 replaced by 0)
        \item \textit{Pros}: Preserves maximum information; separates ``whether'' from ``when''
        \item \textit{Cons}: Adds one feature; slightly more complex preprocessing
    \end{itemize}
\end{enumerate}

\textbf{Option C} was chosen because it preserves the maximum amount of information from the original variable. The binary flag captures whether a previous contact occurred (which is highly predictive), while the continuous value captures the recency of that contact for clients who were contacted.
% Task 2: Outer and Inner Evaluation
% ============================================================================
\section{Task 2: Outer and Inner Evaluation}
% ============================================================================

\subsection{Evaluation Strategy Overview}

The evaluation follows a two-level approach:
\begin{enumerate}
    \item \textbf{Outer Evaluation (Holdout):} A train/test split to estimate final model performance
    \item \textbf{Inner Evaluation (Cross-Validation):} Used during hyperparameter tuning to avoid overfitting
\end{enumerate}

This nested approach ensures that the test set remains truly unseen until the final evaluation.

\subsection{Target Variable Encoding}

The target variable \texttt{deposit} contains string values ('yes' and 'no'). To ensure compatibility with scikit-learn models (especially K-Nearest Neighbors), these were encoded into numeric values using \texttt{LabelEncoder}:
\begin{itemize}
    \item \texttt{no} $\rightarrow$ 0
    \item \texttt{yes} $\rightarrow$ 1
\end{itemize}

This encoding is performed prior to the train/test split to ensure consistency across all data subsets.

\subsection{Outer Evaluation: Train/Test Split}

\begin{table}[H]
\centering
\caption{Train/Test Split Configuration}
\begin{tabular}{lr}
\toprule
\textbf{Parameter} & \textbf{Value} \\
\midrule
Split ratio & 80\% train / 20\% test \\
Training samples & 8,800 \\
Test samples & 2,200 \\
Stratification & Yes (on target variable) \\
\bottomrule
\end{tabular}
\end{table}

The split uses \texttt{train\_test\_split} with \texttt{stratify=y} to maintain class proportions in both sets:

\begin{table}[H]
\centering
\caption{Class Distribution After Split}
\begin{tabular}{lrrrr}
\toprule
 & \multicolumn{2}{c}{\textbf{Training Set}} & \multicolumn{2}{c}{\textbf{Test Set}} \\
\textbf{Class} & Count & \% & Count & \% \\
\midrule
no & 4,624 & 52.55\% & 1,156 & 52.55\% \\
yes & 4,176 & 47.45\% & 1,044 & 47.45\% \\
\bottomrule
\end{tabular}
\end{table}

\subsection{Inner Evaluation Strategy}

For hyperparameter tuning (Task 3), \texttt{StratifiedKFold} cross-validation is used:

\begin{table}[H]
\centering
\caption{Cross-Validation Configuration}
\begin{tabular}{lr}
\toprule
\textbf{Parameter} & \textbf{Value} \\
\midrule
Strategy & StratifiedKFold \\
Number of folds & 3 \\
Shuffle & True \\
\bottomrule
\end{tabular}
\end{table}

\textbf{Justification for StratifiedKFold:} While the dataset is fairly balanced, stratification ensures each fold maintains the same class proportions, leading to more reliable performance estimates during hyperparameter tuning.

\subsection{Preprocessing Pipeline}

All preprocessing is implemented using scikit-learn \texttt{Pipeline} and \texttt{ColumnTransformer} to ensure proper fit/transform separation and prevent data leakage.

\begin{table}[H]
\centering
\caption{Preprocessing Steps by Feature Type}
\begin{tabular}{lll}
\toprule
\textbf{Feature Type} & \textbf{Imputation} & \textbf{Transformation} \\
\midrule
Numerical (6 features) & Median & StandardScaler \\
Categorical (10 features) & Most frequent & OneHotEncoder \\
pdays (special) & Custom & Binary flag + scaled continuous \\
\bottomrule
\end{tabular}
\end{table}

\subsubsection{Custom pdays Preprocessing}

A custom \texttt{PdaysTransformer} was implemented to handle the \texttt{pdays} variable:

\begin{lstlisting}[language=Python, caption=PdaysTransformer (Option C)]
class PdaysTransformer(BaseEstimator, TransformerMixin):
    def fit(self, X, y=None):
        return self  # Stateless transformation
    
    def transform(self, X):
        was_contacted = (X != -1).astype(int)
        pdays_transformed = np.where(X == -1, 0, X)
        return np.column_stack([was_contacted, 
                                 pdays_transformed])
\end{lstlisting}

This creates two features from the original \texttt{pdays}:
\begin{itemize}
    \item \texttt{was\_contacted\_before}: Binary indicator (1 if previously contacted)
    \item \texttt{pdays\_transformed}: Days since contact (with -1 replaced by 0)
\end{itemize}

\subsubsection{Feature Expansion}

After preprocessing, the feature space expands due to one-hot encoding:

\begin{table}[H]
\centering
\caption{Feature Dimensions}
\begin{tabular}{lr}
\toprule
\textbf{Stage} & \textbf{Dimensions} \\
\midrule
Original features & 16 \\
After preprocessing & 53 \\
\bottomrule
\end{tabular}
\end{table}

\subsection{Evaluation Metrics}

\subsubsection{Main Metric: Accuracy}
Accuracy is appropriate as the main metric because:
\begin{itemize}
    \item The classes are balanced (52.5\% vs 47.5\%)
    \item It provides an intuitive interpretation of model performance
    \item It is widely used for comparison in classification tasks
\end{itemize}

\subsubsection{Secondary Metric: Confusion Matrices}
Confusion matrices provide detailed insight into:
\begin{itemize}
    \item True positives and true negatives
    \item Type I errors (false positives) and Type II errors (false negatives)
    \item Potential class-specific performance issues
\end{itemize}
% Task 3: Decision Trees and KNN
% ============================================================================
\section{Task 3: Decision Trees and KNN Comparison}
% ============================================================================

This section presents the training, evaluation, and comparison of Decision Tree and K-Nearest Neighbors classifiers, along with hyperparameter optimization.

\subsection{Baseline: Dummy Classifier}

A Dummy classifier using the ``most\_frequent'' strategy provides a baseline that any real model should beat. This classifier always predicts the majority class.

\begin{table}[H]
\centering
\caption{Dummy Classifier Results}
\begin{tabular}{lr}
\toprule
\textbf{Metric} & \textbf{Value} \\
\midrule
Strategy & Most frequent \\
Accuracy & 52.55\% \\
Training time & 0.03s \\
\bottomrule
\end{tabular}
\end{table}

\begin{table}[H]
\centering
\caption{Confusion Matrix: Dummy Classifier}
\begin{tabular}{lcc}
\toprule
& \textbf{Predicted No} & \textbf{Predicted Yes} \\
\midrule
\textbf{Actual No} & 1156 & 0 \\
\textbf{Actual Yes} & 1044 & 0 \\
\bottomrule
\end{tabular}
\end{table}

The baseline accuracy of 52.55\% reflects the majority class proportion, confirming the balanced dataset analysis from Task 1.

\subsection{Decision Trees}

\subsubsection{Default Hyperparameters}

Decision Tree with scikit-learn's default parameters:

\begin{table}[H]
\centering
\caption{Decision Tree (Default) Results}
\begin{tabular}{lr}
\toprule
\textbf{Metric} & \textbf{Value} \\
\midrule
Accuracy & 79.86\% \\
Training time & 0.09s \\
\bottomrule
\end{tabular}
\end{table}

\begin{table}[H]
\centering
\caption{Confusion Matrix: Decision Tree (Default)}
\begin{tabular}{lcc}
\toprule
& \textbf{Predicted No} & \textbf{Predicted Yes} \\
\midrule
\textbf{Actual No} & 954 & 202 \\
\textbf{Actual Yes} & 241 & 803 \\
\bottomrule
\end{tabular}
\end{table}

\subsubsection{Hyperparameter Tuning}

GridSearchCV was used to tune the Decision Tree with the following parameter grid:

\begin{lstlisting}[language=Python, caption=Decision Tree Parameter Grid]
param_grid = {
    'max_depth': [3, 5, 10, 15, None],
    'min_samples_split': [2, 5, 10],
    'min_samples_leaf': [1, 2, 4],
    'criterion': ['gini', 'entropy']
}
# Total: 90 combinations x 3 folds = 270 fits
\end{lstlisting}

\begin{table}[H]
\centering
\caption{Decision Tree (Tuned) Results}
\begin{tabular}{lr}
\toprule
\textbf{Metric} & \textbf{Value} \\
\midrule
Test Accuracy & 82.95\% \\
CV Mean (±std) & 81.44\% (±0.18\%) \\
HPO Time & 4.27s \\
\midrule
\texttt{max\_depth} & 10 \\
\texttt{min\_samples\_split} & 5 \\
\texttt{min\_samples\_leaf} & 1 \\
\texttt{criterion} & gini \\
\bottomrule
\end{tabular}
\end{table}

\begin{table}[H]
\centering
\caption{Confusion Matrix: Decision Tree (Tuned)}
\begin{tabular}{lcc}
\toprule
& \textbf{Predicted No} & \textbf{Predicted Yes} \\
\midrule
\textbf{Actual No} & 984 & 172 \\
\textbf{Actual Yes} & 203 & 841 \\
\bottomrule
\end{tabular}
\end{table}

\subsection{K-Nearest Neighbors (KNN)}

\subsubsection{Pipeline Construction}

KNN requires preprocessing through a pipeline that includes:
\begin{itemize}
    \item Numerical features: Imputation (median) + Scaling
    \item Categorical features: Imputation (most frequent) + One-hot encoding
    \item pdays: Custom transformer (binary flag + transformed continuous)
\end{itemize}

\subsubsection{Scaling Method Comparison}

Two scaling methods were compared using 3-fold cross-validation:

\begin{table}[H]
\centering
\caption{Scaler Comparison for KNN}
\begin{tabular}{lrrr}
\toprule
\textbf{Scaler} & \textbf{CV Accuracy} & \textbf{CV Std} & \textbf{CV Time} \\
\midrule
StandardScaler & \textbf{79.80\%} & ±0.67\% & 1.84s \\
MinMaxScaler & 73.74\% & ±0.01\% & 1.18s \\
\bottomrule
\end{tabular}
\end{table}

\textbf{Selected scaler: StandardScaler} (+6.06 percentage points over MinMaxScaler)

StandardScaler significantly outperforms MinMaxScaler for this dataset. This is likely because:
\begin{itemize}
    \item StandardScaler is more robust to outliers in the numerical features
    \item The z-score normalization better preserves the relative distances important for KNN
\end{itemize}

\subsubsection{Default Hyperparameters}

KNN with default parameters (k=5) and StandardScaler:

\begin{table}[H]
\centering
\caption{KNN (Default) Results}
\begin{tabular}{lr}
\toprule
\textbf{Metric} & \textbf{Value} \\
\midrule
Accuracy & 81.18\% \\
Training time & 0.03s \\
Scaler & StandardScaler \\
\bottomrule
\end{tabular}
\end{table}

\begin{table}[H]
\centering
\caption{Confusion Matrix: KNN (Default)}
\begin{tabular}{lcc}
\toprule
& \textbf{Predicted No} & \textbf{Predicted Yes} \\
\midrule
\textbf{Actual No} & 985 & 171 \\
\textbf{Actual Yes} & 243 & 801 \\
\bottomrule
\end{tabular}
\end{table}

\subsubsection{Hyperparameter Tuning}

GridSearchCV was used to tune KNN:

\begin{lstlisting}[language=Python, caption=KNN Parameter Grid]
param_grid = {
    'n_neighbors': [3, 5, 7, 11, 15, 21],
    'weights': ['uniform', 'distance'],
    'metric': ['euclidean', 'manhattan', 'minkowski']
}
# Total: 36 combinations x 3 folds = 108 fits
\end{lstlisting}

\begin{table}[H]
\centering
\caption{KNN (Tuned) Results}
\begin{tabular}{lr}
\toprule
\textbf{Metric} & \textbf{Value} \\
\midrule
Test Accuracy & 82.18\% \\
CV Mean (±std) & 81.06\% (±0.38\%) \\
HPO Time & 11.75s \\
\midrule
\texttt{n\_neighbors} & 11 \\
\texttt{weights} & distance \\
\texttt{metric} & euclidean \\
\bottomrule
\end{tabular}
\end{table}

\begin{table}[H]
\centering
\caption{Confusion Matrix: KNN (Tuned)}
\begin{tabular}{lcc}
\toprule
& \textbf{Predicted No} & \textbf{Predicted Yes} \\
\midrule
\textbf{Actual No} & 997 & 159 \\
\textbf{Actual Yes} & 233 & 811 \\
\bottomrule
\end{tabular}
\end{table}

\subsection{Comparison Results}

\begin{table}[H]
\centering
\caption{Model Comparison Summary}
\begin{tabular}{lrrr}
\toprule
\textbf{Model} & \textbf{Test Accuracy} & \textbf{CV Mean} & \textbf{Train Time} \\
\midrule
Dummy (baseline) & 52.55\% & -- & 0.03s \\
Decision Tree (default) & 79.86\% & -- & 0.09s \\
\textbf{Decision Tree (tuned)} & \textbf{82.95\%} & 81.44\% & 4.27s \\
KNN (default) & 81.18\% & -- & 0.03s \\
KNN (tuned) & 82.18\% & 81.06\% & 11.75s \\
\bottomrule
\end{tabular}
\end{table}

\subsection{Summary}

\begin{itemize}
    \item All models significantly outperform the dummy baseline
    \item StandardScaler is the preferred scaler for KNN (+6.06\% over MinMaxScaler)
    \item Hyperparameter optimization improves both models
    \item \textbf{Best model:} Decision Tree (tuned) at 82.95\%, outperforming KNN (tuned) at 82.18\%
    \item Decision Tree is preferred for Task 4 due to better accuracy, faster training, and interpretability
\end{itemize}

% Task 4: Results and Final Model
% ============================================================================
\section{Task 4: Results and Final Model}
% ============================================================================

This section details the selection of the best model, its evaluation on the test set, and the generation of predictions for the competition dataset.

\subsection{Best Model Selection}

Based on the comparison in Task 3, the \textbf{Decision Tree (Tuned)} was selected as the final model.

\begin{table}[H]
\centering
\caption{Selected Model: Decision Tree (Tuned)}
\begin{tabular}{lr}
\toprule
\textbf{Metric} & \textbf{Value} \\
\midrule
Test Accuracy & 82.95\% \\
Training Time & 4.27s \\
Complexity & Moderate (Depth=10) \\
\bottomrule
\end{tabular}
\end{table}

The Decision Tree was chosen over KNN because:
\begin{enumerate}
    \item \textbf{Higher Accuracy:} Decision Tree (82.95\%) outperforms KNN (82.18\%).
    \item \textbf{Efficiency:} It trains and optimizes significantly faster (4.27s vs 11.75s for HPO).
    \item \textbf{Inference Speed:} Decision Trees provide faster predictions than KNN, which scales linearly with dataset size.
    \item \textbf{Interpretability:} The tree structure allows for easier explanation of decision rules compared to distance-based neighbors.
\end{enumerate}

\subsection{Outer Evaluation (Test Set)}

The outer evaluation on the held-out test set (2,200 samples) confirms the model's ability to generalize:

\begin{itemize}
    \item \textbf{Accuracy:} 82.95\%
    \item \textbf{Performance vs Baseline:} +30.40 percentage points over the Dummy classifier (52.55\%).
\end{itemize}

\begin{table}[H]
\centering
\caption{Confusion Matrix: Final Model (Test Set)}
\begin{tabular}{lcc}
\toprule
& \textbf{Predicted No} & \textbf{Predicted Yes} \\
\midrule
\textbf{Actual No} & 984 & 172 \\
\textbf{Actual Yes} & 203 & 841 \\
\bottomrule
\end{tabular}
\end{table}

\subsection{Final Model Training}

To maximize performance for the competition, the final model was retrained on the full dataset (combining training and test sets).

\begin{table}[H]
\centering
\caption{Final Feature Space}
\begin{tabular}{lr}
\toprule
\textbf{Parameter} & \textbf{Value} \\
\midrule
Total Instances & 11,000 \\
Original Features & 16 \\
Transformed Features & 53 \\
\bottomrule
\end{tabular}
\end{table}

The final pipeline consists of:
\begin{enumerate}
    \item \textbf{Preprocessing:} ColumnTransformer (Imputation, Scaling, Encoding, Pdays transformation).
    \item \textbf{Classifier:} DecisionTreeClassifier with optimized hyperparameters:
    \begin{itemize}
        \item \texttt{criterion='entropy'}
        \item \texttt{max\_depth=10}
        \item \texttt{min\_samples\_split=10}
        \item \texttt{min\_samples\_leaf=2}
    \end{itemize}
\end{enumerate}

\subsubsection{Decision Tree Visualization}

One of the key advantages of Decision Trees is their interpretability. Figure~\ref{fig:decision_tree} shows the top levels of the trained decision tree, illustrating how the model makes classification decisions.

\begin{figure}[H]
\centering
\includegraphics[width=\textwidth]{figures/decision_tree_simple.png}
\caption{Decision Tree Structure (Top 3 Levels). The tree uses features like \texttt{duration} (call duration) and \texttt{poutcome} (previous campaign outcome) as primary splitting criteria. Colors indicate class predictions (orange for ``no'', blue for ``yes'').}
\label{fig:decision_tree}
\end{figure}

The full tree has a depth of 10 with 245 leaf nodes, trained on 53 transformed features. The visualization demonstrates how the model can be interpreted to understand which features drive predictions.

\subsection{Competition Predictions}

Predictions were generated for the \texttt{bank\_competition.pkl} dataset (162 instances) using the retrained final model.

\begin{itemize}
    \item \textbf{Input:} 162 samples, 16 features.
    \item \textbf{Output:} 162 predictions mapped to original 'yes'/'no' labels.
    \item \textbf{File:} \texttt{outputs/competition\_predictions.csv}
\end{itemize}

The predictions follow the required format (Id, deposit). A sample of the output is shown below:

\begin{table}[H]
\centering
\caption{Sample Competition Predictions}
\resizebox{\textwidth}{!}{
\begin{tabular}{lrlllrrrl}
\toprule
\textbf{Id} & \textbf{Age} & \textbf{Job} & \textbf{Marital} & \textbf{Balance} & \textbf{Pdays} & \textbf{Prediction} \\
\midrule
5553 & 43 & management & married & 78 & 109 & no \\
915 & 34 & housemaid & married & 0 & -1 & yes \\
7652 & 54 & technician & married & 3323 & -1 & no \\
5065 & 43 & blue-collar & single & -399 & -1 & no \\
3338 & 35 & blue-collar & married & 262 & 181 & yes \\
\bottomrule
\end{tabular}
}
\end{table}

% Task 5: Open-Choice Task
% ============================================================================
\section{Task 5: Open-Choice Task (CatBoost)}
% ============================================================================

For the open-choice task, \textbf{CatBoost} was selected, a gradient boosting library developed by Yandex.

\subsection{Justification}

The selection of CatBoost was driven by the nature of the dataset's features:
\begin{enumerate}
    \item \textbf{Categorical Data Handling:} The dataset contains 10 categorical variables, including \texttt{job} (12 levels) and \texttt{month} (12 levels). Standard One-Hot Encoding (used in Task 3/4) increases dimensionality significantly (from 16 to 53 features). CatBoost handles categorical features natively using techniques like Ordered Target Statistics, which often preserves more information and reduces sparsity.
    \item \textbf{Gradient Boosting:} Gradient Boosted Decision Trees (GBDT) generally outperform single Decision Trees (Task 4) by iteratively correcting errors of previous trees.
    \item \textbf{Robustness:} CatBoost is known for performing well with default hyperparameters, reducing the need for extensive tuning.
\end{enumerate}

\subsection{Implementation}

Unlike the scikit-learn pipeline used in previous tasks, the implementation for CatBoost involved:
\begin{itemize}
    \item \textbf{No One-Hot Encoding:} Categorical features were passed directly to the model as raw strings/objects.
    \item \textbf{Pdays Transformation:} The same 2-feature transformation logic for \texttt{pdays} was applied manually to untangle the '-1' value.
    \item \textbf{Missing Values:} \texttt{job} missing values were explicitly filled with 'unknown'.
\end{itemize}

The model was trained with the following configuration:
\begin{itemize}
    \item \textbf{Iterations:} 1000 (Early stopping at 50 rounds)
    \item \textbf{Depth:} 6
    \item \textbf{Learning Rate:} 0.05
    \item \textbf{Loss Function:} Logloss (optimized for Accuracy)
\end{itemize}

\subsection{Results Evaluation}

The CatBoost model was compared against the tuned Decision Tree from Task 4.

\begin{table}[H]
\centering
\caption{Task 5: CatBoost Performance vs Baseline}
\begin{tabular}{lrrr}
\toprule
\textbf{Metric} & \textbf{Decision Tree (Tuned)} & \textbf{CatBoost (Default)} & \textbf{Improvement} \\
\midrule
Accuracy & 82.95\% & \textbf{87.23\%} & +4.28\% \\
Training Time & 4.27s & 7.41s & -3.14s \\
\bottomrule
\end{tabular}
\end{table}

\begin{table}[H]
\centering
\caption{Confusion Matrix: CatBoost}
\begin{tabular}{lcc}
\toprule
& \textbf{Predicted No} & \textbf{Predicted Yes} \\
\midrule
\textbf{Actual No} & 991 & 165 \\
\textbf{Actual Yes} & 116 & 928 \\
\bottomrule
\end{tabular}
\end{table}

\paragraph{Analysis}
The CatBoost model achieved a significant improvement over the best model from previous tasks.
\begin{itemize}
    \item \textbf{Accuracy:} Increased from 82.95\% to 87.23\%.
    \item \textbf{Recall (Positive Class):} The most notable improvement is in detecting successful deposits. The Decision Tree correctly identified 841 positives, while CatBoost identified 928 (an increase of 87 true positives).
    \item \textbf{False Negatives:} Missed opportunities dropped from 203 to 116.
\end{itemize}

\subsection{Conclusion}

The selection of CatBoost was highly effective. The native handling of categorical features properly exploited the information in high-cardinality variables like 'job' and 'month' without the dilution caused by massive one-hot encoding. This demonstrates that for tabular data with mixed feature types, gradient boosting with specialized categorical handling is superior to single decision tree baselines.

% Task 6: Feature Selection (Optional)
% ============================================================================
\section{Task 6: Feature Selection for KNN (Optional)}
% ============================================================================

In this optional task, it was explored whether the dimensionality of the dataset could be reduced without compromising the performance of the KNN model. The \texttt{SelectKBest} method combined with GridSearchCV was used to determine the optimal number of features.

\subsection{Methodology}

\begin{itemize}
    \item \textbf{Feature Selection Method:} \texttt{SelectKBest} with the ANOVA F-value (\texttt{f\_classif}) scoring function.
    \item \textbf{Model:} K-Nearest Neighbors Classifier with the optimal hyperparameters found in Task 3 ($k=11$, weights='distance', metric='euclidean').
    \item \textbf{Search Strategy:} Grid Search over the parameter $k$ (number of features) ranging from 5 to 53.
    \item \textbf{Preprocessing:} Standard One-Hot Encoding pipeline used in Task 2.
\end{itemize}

\subsection{Results}

The Grid Search identified the following optimal configuration:

\begin{itemize}
    \item \textbf{Optimal Number of Features:} 53 (out of 53 total features).
    \item \textbf{Test Accuracy with Selected Features:} 82.18\%
\end{itemize}

\paragraph{Analysis}
The feature selection process determined that all 53 features should be retained:
\begin{itemize}
    \item \textbf{Dimensionality Reduction:} No reduction was achieved. The full feature set was identified as optimal by the grid search.
    \item \textbf{Performance:} The test accuracy of 82.18\% is consistent with the full KNN model results from Task 3, as expected given that no features were removed.
\end{itemize}

\subsection{Conclusion}
The feature selection process found that all features contribute to the KNN model's performance. This indicates that the preprocessing and feature engineering steps in Task 2 did not introduce redundant or noisy features relative to the distance-based learning of KNN.


\end{document}
