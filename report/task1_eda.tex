% Task 1: Simplified Exploratory Data Analysis
% ============================================================================
\section{Task 1: Simplified Exploratory Data Analysis}
% ============================================================================

This section presents the results of a simplified Exploratory Data Analysis (EDA) performed on the Bank Marketing dataset.

\subsection{Dataset Overview}

The Bank Marketing dataset contains information about direct marketing campaigns (phone calls) of a Portuguese banking institution. The goal is to predict whether a client will subscribe to a term deposit.

\begin{table}[H]
\centering
\caption{Dataset Dimensions}
\begin{tabular}{lr}
\toprule
\textbf{Metric} & \textbf{Value} \\
\midrule
Number of instances & 11,000 \\
Number of features & 16 \\
Target variable & \texttt{deposit} \\
\bottomrule
\end{tabular}
\end{table}

\subsection{Variable Types}

Variables were classified based on their data types. Columns with \texttt{object} type are considered categorical, while numeric types (\texttt{int64}, \texttt{float64}) are considered numerical.

\subsubsection{Numerical Variables (7)}
\begin{itemize}
    \item \texttt{age} -- age in years
    \item \texttt{balance} -- average yearly balance
    \item \texttt{day} -- last contact day
    \item \texttt{duration} -- last contact duration in seconds
    \item \texttt{campaign} -- number of contacts during this campaign
    \item \texttt{pdays} -- days since last contact from previous campaign
    \item \texttt{previous} -- number of contacts before this campaign
\end{itemize}

\subsubsection{Categorical Variables (10)}
\begin{itemize}
    \item \texttt{job} -- type of job (12 unique values)
    \item \texttt{marital} -- marital status (3 unique values)
    \item \texttt{education} -- education level (4 unique values)
    \item \texttt{default} -- has credit in default (2 unique values)
    \item \texttt{housing} -- has housing loan (2 unique values)
    \item \texttt{loan} -- has personal loan (2 unique values)
    \item \texttt{contact} -- contact communication type (3 unique values)
    \item \texttt{month} -- last contact month (12 unique values)
    \item \texttt{poutcome} -- outcome of previous campaign (4 unique values)
    \item \texttt{deposit} -- target variable (2 unique values)
\end{itemize}

\subsubsection{High Cardinality Variables}

Using a threshold of 10 unique values, the following categorical variables exhibit high cardinality:

\begin{table}[H]
\centering
\caption{High Cardinality Categorical Variables}
\begin{tabular}{lr}
\toprule
\textbf{Variable} & \textbf{Unique Values} \\
\midrule
\texttt{job} & 12 \\
\texttt{month} & 12 \\
\bottomrule
\end{tabular}
\end{table}

These variables may require special encoding strategies (e.g., target encoding) in preprocessing, though standard one-hot encoding remains viable given the moderate cardinality.

\subsection{Missing Values Analysis}

\begin{table}[H]
\centering
\caption{Columns with Missing Values}
\begin{tabular}{lrr}
\toprule
\textbf{Column} & \textbf{Missing Count} & \textbf{Percentage} \\
\midrule
\texttt{job} & 327 & 2.97\% \\
\bottomrule
\end{tabular}
\end{table}

Only the \texttt{job} column contains missing values, representing approximately 3\% of the data. This will require imputation during preprocessing, where the most frequent category or a separate ``missing'' category can be used.

\subsection{Constant and ID Columns}

The analysis checked for:
\begin{itemize}
    \item \textbf{Constant columns}: Columns where all values are identical (useless for prediction)
    \item \textbf{Potential ID columns}: Columns where all values are unique (likely identifiers)
\end{itemize}

\textbf{Result}: No constant columns and no potential ID columns were found in the dataset. All features contain meaningful variation.

\subsection{Problem Type Identification}

\begin{table}[H]
\centering
\caption{Target Variable Analysis}
\begin{tabular}{lr}
\toprule
\textbf{Property} & \textbf{Value} \\
\midrule
Target variable & \texttt{deposit} \\
Data type & object (categorical) \\
Unique values & 2 (yes, no) \\
Problem type & \textbf{Binary Classification} \\
\bottomrule
\end{tabular}
\end{table}

\subsection{Class Imbalance}

\begin{table}[H]
\centering
\caption{Class Distribution}
\begin{tabular}{lrr}
\toprule
\textbf{Class} & \textbf{Count} & \textbf{Percentage} \\
\midrule
no & 5,780 & 52.55\% \\
yes & 5,220 & 47.45\% \\
\midrule
\textbf{Imbalance ratio} & \multicolumn{2}{c}{1.11} \\
\bottomrule
\end{tabular}
\end{table}

The dataset is \textbf{not imbalanced}. With an imbalance ratio of 1.11 (using a threshold of 1.5), the classes are well-balanced. This means we can use accuracy as an appropriate evaluation metric without requiring techniques like oversampling or class weighting.

\subsection{Special Variable: pdays}

The \texttt{pdays} variable requires special attention as it encodes two types of information:
\begin{itemize}
    \item \textbf{-1}: Client was not contacted in a previous campaign (or unknown)
    \item \textbf{$\geq$ 0}: Number of days since the client was last contacted
\end{itemize}

\begin{table}[H]
\centering
\caption{pdays Analysis}
\begin{tabular}{lr}
\toprule
\textbf{Metric} & \textbf{Value} \\
\midrule
Total observations & 11,000 \\
Values = -1 (no previous contact) & 8,203 (74.57\%) \\
Values $>$ -1 (previous contact) & 2,797 (25.43\%) \\
\midrule
\multicolumn{2}{c}{\textit{Statistics for contacted clients only:}} \\
Minimum & 1 days \\
Maximum & 854 days \\
Mean & 204.72 days \\
Median & 182 days \\
\bottomrule
\end{tabular}
\end{table}

\subsubsection{Preprocessing Alternatives}

Three preprocessing strategies were considered for handling \texttt{pdays}:

\begin{enumerate}
    \item \textbf{Option A: Median Imputation}
    \begin{itemize}
        \item Replace -1 values with the median of valid pdays values (182 days)
        \item \textit{Pros}: Simple, maintains single feature
        \item \textit{Cons}: Loses information about whether contact occurred; may introduce misleading patterns
    \end{itemize}
    
    \item \textbf{Option B: Large Value Replacement}
    \begin{itemize}
        \item Replace -1 with a large value (e.g., 999 or max+1 = 855)
        \item \textit{Pros}: Simple, maintains single feature, distinguishes no-contact cases
        \item \textit{Cons}: Arbitrary choice of replacement value; may affect distance-based methods
    \end{itemize}
    
    \item \textbf{Option C: Two-Feature Approach (Selected)}
    \begin{itemize}
        \item Create binary indicator: \texttt{was\_contacted\_before} = 1 if pdays $\neq$ -1, else 0
        \item Keep transformed \texttt{pdays} column (with -1 replaced by 0)
        \item \textit{Pros}: Preserves maximum information; separates ``whether'' from ``when''
        \item \textit{Cons}: Adds one feature; slightly more complex preprocessing
    \end{itemize}
\end{enumerate}

\textbf{Option C} was chosen because it preserves the maximum amount of information from the original variable. The binary flag captures whether a previous contact occurred (which is highly predictive), while the continuous value captures the recency of that contact for clients who were contacted.